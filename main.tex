\documentclass[12pt]{amsart}
\usepackage{amsmath}
\usepackage{amsthm}
\usepackage{amsfonts}
\usepackage{amssymb}
\usepackage[margin=1in]{geometry}
\usepackage{stackengine}
\usepackage{hyperref}
\hypersetup{
    colorlinks=true,
    linkcolor=blue
}

\theoremstyle{definition}
\newtheorem{theorem}{Theorem}[section]
\newtheorem{lemma}[theorem]{Lemma}
\newtheorem{definition}[theorem]{Definition}
\newtheorem{corollary}[theorem]{Corollary}
\newtheorem{proposition}[theorem]{Proposition}
\newtheorem{conjecture}[theorem]{Conjecture}
\newtheorem{remark}[theorem]{Remark}
\newtheorem{example}[theorem]{Example}
\newtheorem{problem}[theorem]{Problem}
\newtheorem{notation}[theorem]{Notation}
\newtheorem{question}[theorem]{Question}
\newtheorem{caution}[theorem]{Caution}

\begin{document}

\title{Homework}

\maketitle

For this week, please answer the following questions from the text. 
I've copied the problem itself below and the question numbers for 
your convenience. 

\begin{enumerate}
	\item (6.3) Suppose that the cubic polynomial $X^3 + AX + B$ factors 
		as 
	\begin{displaymath}
		X^3 + AX + B = (X-e_1)(X-e_2)(X-e_3)
	\end{displaymath}
		Prove that $4A^3+27B^2 = 0$ if and only if two (or more) of 
		$e_1,e_2,$ and $e_3$ are the same. (Hint: Multiply out the 
		right-hand side and compare coefficients to relate $A$ and 
		and $B$ to $e_1,e_2,e_3.$) 

		Suppose that instead we start with the cubic $X^3 + AX^2 + 
		BX + C$. What is the formula, in terms of $A, B$, and $C$ 
		for its discriminant? 

	\item (6.6) Make an addition table for $E$ over $\mathbb{F}_p$, 
		as we did in Table 6.1. 
	\begin{enumerate}
		\item $E: Y^2 = X^3 + X + 2$ over $\mathbb{F}_5$.
		\item $E: Y^2 = X^3 + 2X + 3$ over $\mathbb{F}_7$.
		\item $E: Y^2 = X^3 + 2X + 5$ over $\mathbb{F}_{11}$.
	\end{enumerate}

	\item (6.9) Let $E$ be an elliptic curve over $\mathbb{F}_p$ and let 
		$P$ and $Q$ be points in $E(\mathbb{F}_p)$. Assume that $Q$ 
		is a multiple of $P$ and let $n_0 > 0$ be the smallest solution 
		to $Q = nP$. Also let $s>0$ be the smallest solution to $sP = 
		\mathcal O$. Prove that every solution to $Q = nP$ looks like 
		$n_0 + is$ for $i \in \mathbb{Z}$. (Hint: Write $n = is + r$ for 
		some $0 \leq r < s$ and determine the value of $r$.)

	\item (6.10) Let $\lbrace P_1,P_2 \rbrace$ be a basis for $E[m]$. The 
		\textit{Basis Problem} for $\lbrace P_1,P_2 \rbrace$ is to 
		express an arbitrary point $P \in E[m]$ as a linear combination 
		of the basis vectors, i.e, to find $n_1$ and $n_2$ so that $P = 
		n_1P_1 + n_2P_2$. Prove that an algorithm that solves the 
		basis problem for $\lbrace P_1,P_2 \rbrace$ can be used to 
		solve the ECDLP for points in $E[m]$.
		
	\item (6.11) Use the double-and-add algorithm (Table 6.3) to compute $nP$ 
		in $E(\mathbb{F}_p)$ for each of the following curves and points, 
		as we did in Fig. 6.4. 
	\begin{enumerate}
		\item $E : Y^2 = X^3 + 23X + 13, p = 83, P = (24,14), n = 19$
		\item $E : Y^2 = X^3 + 143X + 367, p = 613, P = (195,9), n = 23$
		\item $E : Y^2 = X^3 + 1828X + 1675, p = 1999, P = (1756,348), n = 11$
		\item $E : Y^2 = X^3 + 1541X + 1335, p = 3221, P = (2998,439), n = 3211$
	\end{enumerate}
\end{enumerate}
\end{document}
